\documentclass[12pt]{article}
\usepackage{amsmath}
\usepackage{amssymb}
\usepackage{graphicx}
\usepackage{hyperref}
\usepackage{geometry}
\geometry{margin=1in}

\title{Final Exam Study Guide}
\author{Your Name}
\date{\today}

\begin{document}


\section*{Using Hadoop and Java}
\begin{itemize}
    \item None
\end{itemize}

\section*{1 Map/Reduce Introduction}
Phases:
\begin{itemize}
    \item 1. Map
    \item 2. Local Sort - data sent from map node to a reduce node is sorted
    \item 3. Local Group (optional) - groups by key
    \item 4. Combine (optional) - combines values with the same key
    \item 5. Partition - decides how output of mapper is to be partitioned 
    to the reducers
    \item 6. Global Sort (merge) - a merge sort is applied at reducer node so all
    data with same key is grouped together
    \item 7. Global Group - for each key, create groups
    \item 8. Reduce - merges everyone with same key to produce a single pair output
    for each key 
\end{itemize}

Map/Reduce stores all intermediate data on disk.

\section*{2 Combiner Functions and Custom Classes}
Problems where to use combiner:
\begin{itemize}
    \item finds all words of size $\geq$ n characters and their \textbf{frequency}
    \item Summing up the values of a key using a SumCountPair class
\end{itemize}

\section*{3 Custom Partitioners, Sorters, and Grouping Comparators}
\begin{itemize}
    \item Custom Partition - define function \texttt{getPartition}; on the natural key
    \item Custom Sort - override \texttt{compareTo} method with sort comparator class;
    if primitive type key then create the custom class; ($>$ 0 (1 $>$ 2), 0 (1 = 2), $<$ 0 (1 $<$ 2))
    \item Custom Grouping Comparator (recs with same key passed as input to single 
    call to the reduce method) - write a \texttt{compare} method
    \item Natural key is what final result is grouped by (subset of composite key)
    \item Composite key is output key of mapper (define sorting order on this)
\end{itemize}

\section*{4 Secondary Sort Example}
\begin{itemize}
    \item None
\end{itemize}

\section*{5 Finding Top N Example}
\begin{itemize}
    \item Single call to the reducer when having a top N problem (read problem); all 
    have same key as NULL
\end{itemize}

\section*{6 Outer Joins and Multiple Jobs}
\begin{itemize}
    \item None
\end{itemize}

\section*{7 Intro to Scala}
\begin{itemize}
    \item var vs val (change vs constant)
    \item fold - takes initial value: \texttt{foldLeft(0)((total, x) => total + x)}
    \item reduce - no initial value: \texttt{reduceLeft((acc, x) => acc + x)}
    \item Tuple starts at 1
    \item \_ in a map function is the current element
    \item \_(num) is the num-th element of the tuple (need the \_ to access)
    \item Iterating over maps with case: \texttt{test.map(\{case (k, v) =$>$ println(k + " " + v)\})}
    \item match and case: \texttt{val result = x match \{case 0 =$>$ "zero"; case 1 =$>$ "one"\}}
\end{itemize}

\section*{8 RDDs in Spark}
\begin{itemize}
    \item flatMap - RDD of lists to RDD of elements; example input: \texttt{List(List(1, 2), List(3, 4))} output: \texttt{List(1, 2, 3, 4)};
    \texttt{flatMap(\_.split(" "))} - splits each line into words
    \item saveAsTextFile - saves RDD to a file
    \item union, intersection, subtract, cartesian; cartesian is combinations of two RDDs
    \item collect, count, countByValue, take, top, takeOrdered, reduce, fold, foreach, aggregate
    \item aggregate - \texttt{aggregate((0,0)(seqOp, combOp))} - seqOp is applied to each element in the partition and combOp is applied to the results of seqOp;
    ex. \texttt{rdd.aggregate((0,0))((acc, value) =$>$ (acc.\_1 + value, acc.\_2 + 1), (acc1, acc2) =$>$ (acc1.\_1 + acc2.\_1, acc1.\_2 + acc2.\_2))}
\end{itemize}

\section*{9 Working with Key/Value Pairs}
\begin{itemize}
    \item reduceByKey((x, y) =$>$ x + y) - adds up all values with the same key; combine values with the same key
    \item groupByKey(); returns a list of values for each key
    \item \textbf{mapValues - applies a function to the values of each key}
    \item sortByKey(); sorts by key (true for ascending, false for descending)
    \item keyBy - converts RDD of elements to RDD of key/value pairs; ex \texttt{rdd.keyBy(\_.length)} - adds length as key to each element
    \item sortBy() - sort by a function; ex. \texttt{sortBy(\_.\_(1))}; for descending, use \texttt{sortBy(\_.\_(1), false)}
    \item join - joins two RDDs on the key; ex. \texttt{rdd1.join(rdd2)}; output is (key, (value1, value2))
    \item rightOuterJoin - rdd.join(rdd2) but with all keys in rdd2; if key is not present in rdd, value is None
    \item leftOuterJoin - rdd.join(rdd2) but with all keys in rdd; if key is not present in rdd2, value is None
    \item countByKey - returns a map of key to count
\end{itemize}

\end{document}